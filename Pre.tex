%\documentclass[a4paper,oneside,article]{memoir}
%\documentclass[a4paper,oneside,article]{memoir}
\documentclass[a4paper,oneside,article]{memoir}
\documentclass[a4paper,oneside,article]{memoir}
\include{Pre}



\begin{document}
%-------------------Forside----------------------------------------------
\input{Forside.tex}
%-------------------Indholdsfortenelse-----------------------------------
\renewcommand\contentsname{Indholdsfortegnelse}
\tableofcontents*
\newpage
%--------------------Kapitler--------------------------------------------
\chapter{Introduktion} 
\input{Introduktion.tex}
\newpage
\chapter{- Tre lednings balanceret system}
\input{3-led-balancert.tex}
\newpage
\chapter{- Fire lednings balanceret system}
\input{4-led-balanceret.tex}
\newpage
\chapter{- Tre lednings ubalanceret system}
\input{3-led-ubalancert.tex}
\newpage
\chapter{- Fire lednings ubalanceret system}
\input{4-led-ubalanceret.tex}
\newpage
\chapter{Målinger}
\input{Resultater.tex}
\chapter{Diskussion}
\input{Konklusion.tex}
\end{document}




\begin{document}
%-------------------Forside----------------------------------------------
\begin{titlepage}
   \begin{center}
       \vspace*{3cm}
        
       \Huge
       \textbf{Lab 1 - Modellering af Blackbox}  \\% Title

        \Large
       \vspace{0.5cm}
        IRT E19\\
        \vspace{0.5cm}
        \vspace{9cm}
 
\includegraphics[width=0.4\textwidth]{images/aulogo.jpg} 
 
 
       \vspace{1.5cm}
        \normalsize
        \begin{tabular}{lr}
           \begin{minipage}[t]{7CM} \textbf{Deltagere} \vspace{2mm}\end{minipage}  &  \textbf{AU-ID}\\
            \toprule
            Laurids Vest Thomsen &  AU590472\\
            Steven Petersen      &  AU459931\\
            Christian Moos Lund  &  AU594969\\
            \bottomrule
        \end{tabular}
 
       \vfill

       \vspace{0.8cm}
   \end{center}
\thispagestyle{empty}
\end{titlepage}
\newpage

%-------------------Indholdsfortenelse-----------------------------------
\renewcommand\contentsname{Indholdsfortegnelse}
\tableofcontents*
\newpage
%--------------------Kapitler--------------------------------------------
\chapter{Introduktion} 
\input{Introduktion.tex}
\newpage
\chapter{- Tre lednings balanceret system}
\input{3-led-balancert.tex}
\newpage
\chapter{- Fire lednings balanceret system}
\input{4-led-balanceret.tex}
\newpage
\chapter{- Tre lednings ubalanceret system}
\input{3-led-ubalancert.tex}
\newpage
\chapter{- Fire lednings ubalanceret system}
\input{4-led-ubalanceret.tex}
\newpage
\chapter{Målinger}
\input{Resultater.tex}
\chapter{Diskussion}
\input{Konklusion.tex}
\end{document}




\begin{document}
%-------------------Forside----------------------------------------------
\begin{titlepage}
   \begin{center}
       \vspace*{3cm}
        
       \Huge
       \textbf{Lab 1 - Modellering af Blackbox}  \\% Title

        \Large
       \vspace{0.5cm}
        IRT E19\\
        \vspace{0.5cm}
        \vspace{9cm}
 
\includegraphics[width=0.4\textwidth]{images/aulogo.jpg} 
 
 
       \vspace{1.5cm}
        \normalsize
        \begin{tabular}{lr}
           \begin{minipage}[t]{7CM} \textbf{Deltagere} \vspace{2mm}\end{minipage}  &  \textbf{AU-ID}\\
            \toprule
            Laurids Vest Thomsen &  AU590472\\
            Steven Petersen      &  AU459931\\
            Christian Moos Lund  &  AU594969\\
            \bottomrule
        \end{tabular}
 
       \vfill

       \vspace{0.8cm}
   \end{center}
\thispagestyle{empty}
\end{titlepage}
\newpage

%-------------------Indholdsfortenelse-----------------------------------
\renewcommand\contentsname{Indholdsfortegnelse}
\tableofcontents*
\newpage
%--------------------Kapitler--------------------------------------------
\chapter{Introduktion} 
\input{Introduktion.tex}
\newpage
\chapter{- Tre lednings balanceret system}
\input{3-led-balancert.tex}
\newpage
\chapter{- Fire lednings balanceret system}
\input{4-led-balanceret.tex}
\newpage
\chapter{- Tre lednings ubalanceret system}
\input{3-led-ubalancert.tex}
\newpage
\chapter{- Fire lednings ubalanceret system}
\input{4-led-ubalanceret.tex}
\newpage
\chapter{Målinger}
\input{Resultater.tex}
\chapter{Diskussion}
\input{Konklusion.tex}
\end{document}


% ---------------Preamble fra Latex kursus---------------------
\usepackage[utf8]{inputenc}                 % Korrekt håndtering af æ, ø og å
\usepackage[T1]{fontenc}                    % Korrekt håndtering af æ, ø og å
\usepackage{microtype}                      % Typografisk magi! Giver bl.a. pænere orddeling
\usepackage{graphicx}                       % Gør det muligt at indsætte billeder
\usepackage{amsmath}                        % Giver adgang til uundværlige matematikting
\usepackage{siunitx}                        % Dette  gør alle mat notationer nemmer ! 
\usepackage[danish]{babel}                  % Danske betegnelser og orddeling
\renewcommand{\danishhyphenmins}{22}        % Bedre dansk orddeling

%--------------------Selv fundet Packer------------------------
\usepackage[margin=1.1in]{geometry}         % Marginer
\usepackage{ragged2e}                       % Retter alt ind til venstre
\usepackage{parskip}                        % Strækker ud til begge marginer
\usepackage{booktabs}                       % Tablesgenerator bad om det
\setlength{\parindent}{0pt}                 % Fjerner indents
%--------------------Ekstra------------------------------------
\usepackage[table,xcdraw]{xcolor}
\usepackage{floatrow}
\usepackage{lscape}


%----------------------- Macroer ------------------------------


%----------------------Tabeller--------------------------------
%\toprule               Top tyklinje
%\midrule               Mid tyndlinje
%\bottomrule            Bund tyklinje

%\begin{minipage}[t]{7CM} TEXT \vspace{2mm}\end{minipage}        Tekstombrydning

%-------------------Andre nytte komandoer----------------------
%\plainbreak{x} x = antal linje jeg vil hoppe. 
%\graphicspath{ {./images/} }
%\begin{landscape}
%\end{landscape}
